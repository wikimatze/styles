%% encoding
\usepackage{ngerman}                % germany-settings
\usepackage[utf8]{inputenc}         % Linux character format
\usepackage{color}                  % to define own colors
\definecolor{Dark}{gray}{.5}        % color for cover
\definecolor{Medium}{gray}{.8}      % color for cover
\definecolor{darkred}{rgb}{.5,0,0}  % color for links
\definecolor{darkblue}{rgb}{0,0,.5} % color for links
\definecolor{nb}{gray}{.95}         % color for Def environment
\usepackage{url}                    % needed for hyperref and links in pdfs
  \usepackage[plainpages=false,pdfpagelabels,colorlinks=true,urlcolor=darkblue,pagecolor=darkred,
  citecolor=darkred,linkcolor=darkred]{hyperref} % setting how links are rendered


%% formatting
\parindent1pt                % first row of every chapter will be intended
\columnsep1cm                % no clue
\textwidth38.5em             % width of the text
\textheight23cm              % line height
\setcounter{secnumdepth}{10} % numbering all headings
\setcounter{tocdepth}{4}     % beginning with numbering
\setcounter{section}{0}      % beginning with the counting section


%% various
\usepackage[normalem]{ulem}                                 % uline
\usepackage{array}                                          % for tabulars
\usepackage{longtable}                                      % special tabulars
\usepackage[labelfont=bf]{caption}                          % number of tables automatic bold
\usepackage{csquotes}                                       % quoting technique, very special package with many options
\usepackage{verbatim}                                       % for the \begin{comment}...\end{comment} environment
\usepackage{fancybox}                                       % for the shadowbox
\usepackage{graphicx}                                       % for graphics
\usepackage{float}                                          % for positioning of figures
\DeclareGraphicsRule{.pdftex}{pdf}{.pdftex}{}               % for xfig
\usepackage{wrapfig}                                        % textumflossene Bilder
\usepackage{caption}                                        % caption for images
\usepackage[all]{xy}                                        % image numbering
\usepackage{graphicx}       % for graphics
  \usepackage{float}        % for positioning of figures
  \DeclareGraphicsRule{.pdftex}{pdf}{.pdftex}{} % for xfig
  \usepackage{wrapfig}      % textumflossene Bilder
  \usepackage{caption}      % caption for images

\renewcommand{\thefigure}{\arabic{section}.\arabic{figure}} % numbering for figure environments
\usepackage{pdfpages}                                       % needed for the environments
\usepackage{amsthm}                                         % needed for defining theorem environments
\usepackage{listings}                                       % sourcecode
\usepackage[flushmargin,hang]{footmisc}                     % linebreaks in footnotes will cause no intendation
\usepackage{relsize}                                        % scale the font-size relative
\usepackage{paralist}                                       % own bullet-styling for lists
\usepackage{rotating}                                       % for rotating images
\usepackage{srcltx}                                         % forward/reverse search in dvi
%\usepackage[strings]{underscore} % don't escape underscores

%% layout
\oddsidemargin = 24pt              % margin left
\usepackage{relsize}               % relative font-scaling
\usepackage{mathpazo}              % font-style
\usepackage{fancyhdr, lastpage}    % Anpassung von Kolumnentitel
\fancyhf{}                         % empty the header
\renewcommand{\headrulewidth}{0pt} % width of separating line header
\renewcommand{\footrulewidth}{0pt} % width of separating line footer
\usepackage{picins}                % for floating images


%% new commands
\def\dash{-- }
\def\bab{\vspace{0.3cm}}
\def\ab{\vspace{0.2cm}}
\newcommand{\itbf}[1]{\textbf{\textit{#1}}}
\newcommand{\ulab}[1]{\ab \uline{#1}}
\newcommand{\ulbf}[1]{\textbf{\uline{#1}}}
\newcommand{\ulbfab}[1]{\ab \textbf{\uline{#1}}\ab}
\def\und{u. }
\def\oder{od. }
\newcommand{\kreis}[1]{\textbf{\unitlength1ex\begin{picture}(2.5,2.5)%
\put(0.75,0.75){\circle{2.5}}\put(0.75,0.75){\makebox(0,0){#1}}\end{picture}}}
\newcommand{\frage}[1]{\textcolor{red}{#1\mbox{}\\}}
