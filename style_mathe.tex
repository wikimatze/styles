%% encodings
\usepackage{ngerman}			            % germany-settings
\usepackage[utf8]{inputenc} 		      % Linux Zeichenformat

\usepackage{color}                    % to define own colors
  \definecolor{Dark}{gray}{.5}        % color for cover
  \definecolor{Medium}{gray}{.8}      % color for cover
  \definecolor{darkred}{rgb}{.5,0,0}  % color for links
  \definecolor{darkblue}{rgb}{0,0,.5} % color for links
  \definecolor{nb}{gray}{.95}         % color for Def environment

\usepackage{url}                      % needed for hyperref and links in pdfs
  \usepackage[plainpages=false,pdfpagelabels,colorlinks=true,urlcolor=darkblue,pagecolor=darkred,
  citecolor=darkred,linkcolor=darkred]{hyperref} % setting how links are rendered


%% formatting
\parindent1pt				            % first row of every chapter will be intended
\columnsep1cm				            % no clue
\textwidth38.5em		            % width of the text
\textheight23cm				          % line height
\setcounter{secnumdepth}{10}		% numbering all headings
\setcounter{tocdepth}{4}        % beginning with numbering
\setcounter{section}{0}         % beginning with the counting section


%% math
\usepackage{amsmath}			      % special math environments (\equation, \align, \text, ...)
\usepackage{amsfonts}			      % mathbb und mathfrak
\usepackage{amssymb}			      % greek letters
\usepackage{amsthm}             % needed for defining theorem environments
\usepackage{xspace}			        % for macros with long arrows
\usepackage{pifont}			        % use pi and other symbolfont-fonts		
\usepackage{stmaryrd}           % for \lightning


%% various
\usepackage[normalem]{ulem} % uline
\usepackage{array}			    % for tabulars
\usepackage{longtable}			% special tabulars
\usepackage[labelfont=bf]{caption}	% number of tables automatic bold
\usepackage{csquotes}       % quoting technique, very special package with many options
\usepackage{verbatim}			  % for the \begin{comment}...\end{comment} environment
\usepackage{fancybox}       % for the shadowbox
\usepackage{graphicx}		    % for graphics
  \usepackage{float}			  % for positioning of figures
  \DeclareGraphicsRule{.pdftex}{pdf}{.pdftex}{} % for xfig
  \usepackage{wrapfig}			% textumflossene Bilder
  \usepackage{caption}			% caption for images
\usepackage[all]{xy}        % image numbering
\renewcommand{\thefigure}{\arabic{section}.\arabic{figure}} % numbering for figure environments
\usepackage{pdfpages}       % needed for the environments
\usepackage{listings}			  % sourcecode
\usepackage[flushmargin,hang]{footmisc} % linebreaks in footnotes will cause no intendation
\usepackage{calc, pifont}		% extra symbols
\usepackage{relsize}			  % scale the font-size relative
\usepackage{paralist}			  % own bullet-styling for lists
\usepackage{rotating}       % for rotating images
\usepackage{marvosym}			  % Winkelzeichen
\usepackage{nicefrac}       % better fracs
\usepackage{srcltx}																												% forward/reverse search in dvi
\usepackage[strings]{underscore} % don't escape underscores

\usepackage{array}			% tables
\usepackage{longtable}	% special tables
\usepackage{pifont}			% special symbols (Pi ... and other fonts


%% layout
\oddsidemargin = 24pt			        % margin left
\usepackage{relsize}			        % relative font-scaling
  \usepackage[slantedGreek,sc]{mathpazo}           % font-style
\usepackage{fancyhdr, lastpage}
  \fancyhf{}
  \renewcommand{\headrulewidth}{0pt} % width of separating line header
  \renewcommand{\footrulewidth}{0pt} % width of separating line footer
\usepackage{mathptmx}                 % times new roman font for maththings
\usepackage{picins}               % for floating images


%% new commands
\newcommand{\bab}{\vspace{0.3cm}}
\newcommand{\ab}{\vspace{0.2cm}}
\newcommand{\itbf}[1]{\textbf{\textit{#1}}}
\newcommand{\ulab}[1]{\ab \uline{#1}}
\newcommand{\ulbf}[1]{\textbf{\uline{#1}}}
\newcommand{\ulbfab}[1]{\ab \textbf{\uline{#1}}\ab}
\newcommand{\und}{u. }
\newcommand{\oder}{od. }
\newcommand{\kreis}[1]{\textbf{\unitlength1ex\begin{picture}(2.5,2.5)%
\put(0.75,0.75){\circle{2.5}}\put(0.75,0.75){\makebox(0,0){#1}}\end{picture}}}
\newcommand{\su}{"`}	
\newcommand{\so}{"' }
\newcommand{\hin}{\uline{''$\rightarrow$''}: }
\newcommand{\ruk}{\uline{''$\leftarrow$''}: }
\newcommand{\bin}{\dbinom}
\newcommand{\N}{\ensuremath{\mathbb{N}}\xspace}
\newcommand{\R}{\ensuremath{\mathbb{R}}\xspace}
\newcommand{\C}{\ensuremath{\mathbb{C}}\xspace}
\newcommand{\E}{\ensuremath{\mathbb{E}}\xspace}
\newcommand{\V}{\ensuremath{\mathbb{V}}\xspace}
\newcommand{\Z}{\ensuremath{\mathbb{Z}}\xspace}
\newcommand{\Q}{\ensuremath{\mathbb{Q}}\xspace}
\newcommand{\F}{\ensuremath{\mathbb{F}}\xspace}
\newcommand{\B}{\ensuremath{\mathbb{B}}\xspace}
\renewcommand{\P}{\ensuremath{\mathbb{P}}\xspace}
\newcommand{\HI}{\ensuremath{\mathbb{H}}\xspace}
\newcommand{\ZE}{\ensuremath{\mathfrak{Z}}\xspace}
\newcommand{\SE}{\ensuremath{\mathbb{S}}\xspace}
\newcommand{\AV}{AV }
\newcommand{\UV}{UV }
\newcommand{\ZHG}{ZHG }
\newcommand{\ZV}{ZV }
\newcommand{\NV}{NV }
\newcommand\norm[1]{\ensuremath{\lVert#1\rVert}}
\newcommand\abs[1]{\ensuremath{\lvert#1\rvert}}
\newcommand{\guleft}{\guillemotleft}
\newcommand{\gulright}{\guillemotright}
\renewcommand{\qed}{\hfill \rule{2.5mm}{2.5mm}}		% quod erat demonstrandum
\newcommand\frage[1]{\textcolor{red}{#1\mbox{}\\}}
\def\dash{-- }
